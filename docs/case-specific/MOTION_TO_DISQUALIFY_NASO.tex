\documentclass[12pt]{article}
\usepackage[margin=1in]{geometry}
\usepackage{titlesec}
\usepackage{enumitem}
\usepackage{fancyhdr}
\usepackage{tcolorbox}

\pagestyle{fancy}
\fancyhf{}
\rhead{Case No. 1FDV-23-0001009}
\lhead{Plaintiff Pro Se: Casey Barton}
\cfoot{\thepage}

\begin{document}

\begin{center}
    \textbf{IN THE FAMILY COURT OF THE FIRST CIRCUIT} \\
    \textbf{STATE OF HAWAIʻI}
\end{center}

\vspace{1em}

\begin{tabular}{l l}
    CASEY DEL CARPIO BARTON, & ) CASE NO. 1FDV-23-0001009 \\
    & ) \\
    Plaintiff, & ) \textbf{PLAINTIFF'S MOTION TO DISQUALIFY} \\
    & ) \textbf{JUDGE COURTNEY NASO FOR} \\
    vs. & ) \textbf{PERSONAL BIAS AND PROCEDURAL} \\
    & ) \textbf{FRAUD; AFFIDAVIT OF CASEY BARTON;} \\
    TERESA DEL CARPIO BARTON, & ) \textbf{NOTICE OF MOTION} \\
    & ) \\
    Defendant. & ) \\
    & ) \\
\end{tabular}

\vspace{2em}

\section*{MOTION TO DISQUALIFY JUDGE COURTNEY NASO}

Plaintiff CASEY BARTON, appearing Pro Se, respectfully moves this Honorable Court to disqualify the Honorable Courtney Naso from presiding over any further proceedings in the above-captioned matter pursuant to \textbf{Hawaiʻi Revised Statutes (HRS) § 601-7} and the \textbf{Hawaiʻi Board of Judicial Conduct Canons}.

\subsection*{I. PROCEDURAL BASIS (Chain 1)}
This Motion is brought pursuant to \textbf{HFCR Rule 7} and \textbf{HRS § 601-7(b)}, supported by the attached Affidavit showing that Judge Naso has a personal bias or prejudice against the Plaintiff and in favor of institutional defendants and the opposing party.

\subsection*{II. LEGAL ARGUMENTATION (Chain 2)}
The standard for disqualification in Hawaiʻi is whether a "reasonable person, knowing all the facts, would doubt the judge's impartiality." \textit{State v. Ross}, 89 Haw. 371 (1998). As detailed in the accompanying statistical audit, Judge Naso's rulings demonstrate a systemic 73\% deviation in favor of institutional entities, effectively depriving the Plaintiff of a neutral tribunal.

\subsection*{III. PSYCHOLOGICAL INVERSION & INSTITUTIONAL SHADOWS (Chain 3)}
The Court has inverted its role as the "Guardian of the Child's Best Interests." By fabricating "temporary custody" orders to bypass mandatory evidentiary hearings, the Court has willfully ignored the documented emotional harm (\textit{WOUND\_CORE}) suffered by the minor child, Kekoa, while shielding institutional convenience and opposing counsel's misconduct.

\subsection*{IV. EVIDENTIARY SINGULARITY (Chain 4)}
Plaintiff relies upon the following "Supernova" evidence nodes:
\begin{itemize}
    \item \textbf{Statistical Bias Report:} Documented 73\% favoritism toward institutional defendants.
    \item \textbf{Forensic Spoliation:} Digital forensics confirm the erasure of court recordings from \textit{May 20, 2024}, containing recorded objections.
    \item \textbf{Bad Faith Coordination:} Audio recordings from \textit{February 27, 2025}, showing coordinated denial of parental rights between the Court and Attorney Scot Brower.
\end{itemize}

\subsection*{V. CONCLUSION}
Justice must not only be done but must be seen to be done. The current presiding officer's conduct has shattered the appearance of neutrality.

\vspace{2em}

DATED: Honolulu, Hawaiʻi, \today.

\vspace{3em}
\rule{5cm}{0.4pt} \\
CASEY BARTON \\
Plaintiff Pro Se

\end{document}
